\documentclass{article}

\usepackage[english]{babel}
\usepackage{listings}
\usepackage{color}
\usepackage{lipsum}

\definecolor{dkgreen}{rgb}{0,0.6,0}
\definecolor{gray}{rgb}{0.5,0.5,0.5}
\definecolor{mauve}{rgb}{0.58,0,0.82}

\lstdefinelanguage{JavaScript}{
  keywords={let, typeof, new, true, false, catch, function, return, null, catch, switch, var, if, in, while, do, else, case, break},
  keywordstyle=\color{blue}\bfseries,
  ndkeywords={class, export, boolean, throw, implements, import, this},
  ndkeywordstyle=\color{darkgray}\bfseries,
  identifierstyle=\color{black},
  sensitive=false,
  comment=[l]{//},
  morecomment=[s]{/*}{*/},
  commentstyle=\color{purple}\ttfamily,
  stringstyle=\color{red}\ttfamily,
  morestring=[b]',
  morestring=[b]'
}

\lstset{frame=tb,
  language=JavaScript,
  aboveskip=3mm,
  belowskip=3mm,
  showstringspaces=false,
  columns=flexible,
  basicstyle={\small\ttfamily},
  numbers=none,
  numberstyle=\tiny\color{gray},
  keywordstyle=\color{blue},
  commentstyle=\color{dkgreen},
  stringstyle=\color{mauve},
  breaklines=true,
  breakatwhitespace=true,
  tabsize=3
}

\begin{document}

\section*{Topic 6: Data Link Layer}
Converts digital bits to physical bits for hop-to-hop delivery over a single medium (wired/wireless) \\ \\
Addressing:
\begin{itemize}
    \item Uses MAC address (48-bit globally unique identifier managed by IEEE)
    \item Format: ex. 6E-3F-BB-22-A0-0F
    \item First 3 pairs: Organizational Unique Identifier
    \item Last 3 pairs: Network Interface Card
    \item MAC addresses survive a single hop and can't be used to fool beyond one hop
\end{itemize}

Protocols:
\begin{itemize}
    \item 802.3 Ethernet (wired): simple header and data protocol
    \item 802.11 WiFi (wireless): complex protocol with 100+ message types
    \item Due to different physical properties, protocols differ (WiFi needs authentication, password, etc.)
    \item Ethernet focuses on throughput; WiFi needs complexity for connection handling
\end{itemize}

Header Information:
\begin{itemize}
    \item Preamble: alternating 0s and 1s for synchronization between sender and reciever 
    \item Destination and Source MAC addresses
    \begin{itemize}
        \item Lets us know who the sender/reciever is
    \end{itemize}
    \item EtherType: specifies protocol in payload
    \begin{itemize}
        \item 0x0800 = IPv4, 0x86DD = IPv6, 0x0806 = ARP, 0x8100 = VLAN tagged frame
    \end{itemize}
    \item CRC/FCS: uses CRC32 hash for data integrity
    \item Interframe: prevents collisions
    \item Integrity vs Reliable delivery
    \begin{itemize}
        \item Integrity: detect errors; Reliable delivery: ensure delivery via retransmissions
        \item Ethernet provides integrity, not reliability
    \end{itemize}
\end{itemize}

DLL Devices:
\begin{itemize}
    \item Network Switch (replaced hub):
    \begin{itemize}
        \item Adds CPU, memory for queuing messages
        \item Switch = passive; forwards on behalf of devices
        \item Uses Switch Device Discovery Algorithm:
        \begin{itemize}
            \item Records {SRC MAC + Physical Port} into FIB
            \item If DEST MAC found in FIB, forward on that port
            \item Else fallback to hub mode: forward to all except source port
        \end{itemize}
        \item FIB builds MAC-port mapping over time
        \item Header stays unchanged across switches
        \item Expiry field prevents memory monopolization
        \item Multiple devices can map to one port
    \end{itemize}
\end{itemize}

Management Frames:
\begin{itemize}
        \item Connection/disconnection, 
        \item Beacon: Router constantly emits a beacon sending information for a device
        to connect
        \item Probe: Probing for a network that was connected to previously, to reconnect
        to it
        \item Association: allows us to interact with the network to get network resources or provide resources to other devices through the network
    \end{itemize}
    Control Frames:
    \begin{itemize}
        \item Deals with lossy nature of WiFi 
        \item Acknowledgement: ensures reliable delivery (unlike Ethernet)
        \item Block Ack: acknowledge multiple messages with a single acknowledgement
        \item RTS/CTS: solves hidden node problem; signals other devices to hold transmission
        \begin{itemize}
            \item Ugly solution in contrast to CSMA
            \item Sender sends request to send, Receiver sends clear to send
            \item Signals to devices outside of the senders range to wait and
            switch to RTC + CTS
        \end{itemize}
        \item Costly vs CSMA 
        \begin{itemize}
            \item CSMA is faster, the extra messages cost bandwith and doesn’t
            send actual data
        \end{itemize}
    \end{itemize}

\end{document}