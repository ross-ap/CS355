\documentclass{article}

\usepackage[english]{babel}
\usepackage{listings}
\usepackage{color}
\usepackage{lipsum}

\definecolor{dkgreen}{rgb}{0,0.6,0}
\definecolor{gray}{rgb}{0.5,0.5,0.5}
\definecolor{mauve}{rgb}{0.58,0,0.82}

\lstdefinelanguage{JavaScript}{
  keywords={let, typeof, new, true, false, catch, function, return, null, catch, switch, var, if, in, while, do, else, case, break},
  keywordstyle=\color{blue}\bfseries,
  ndkeywords={class, export, boolean, throw, implements, import, this},
  ndkeywordstyle=\color{darkgray}\bfseries,
  identifierstyle=\color{black},
  sensitive=false,
  comment=[l]{//},
  morecomment=[s]{/*}{*/},
  commentstyle=\color{purple}\ttfamily,
  stringstyle=\color{red}\ttfamily,
  morestring=[b]',
  morestring=[b]'
}

\lstset{frame=tb,
  language=JavaScript,
  aboveskip=3mm,
  belowskip=3mm,
  showstringspaces=false,
  columns=flexible,
  basicstyle={\small\ttfamily},
  numbers=none,
  numberstyle=\tiny\color{gray},
  keywordstyle=\color{blue},
  commentstyle=\color{dkgreen},
  stringstyle=\color{mauve},
  breaklines=true,
  breakatwhitespace=true,
  tabsize=3
}

% Remember to update
\title{Topic 5: Physical Layer}
\date{}
\author{Ross Emile Aparece}

\begin{document}
\maketitle

% Remember to update
\section*{Class 8 \\ 02/25/2025}\label{sec:Class 8}
\begin{itemize}
    \item Physical layer is the lowest level:
    \begin{itemize}
        \item Understanding that the physical properties of various transition mediums define the proctocls we use
        \item Bit by bit encoding of information into physical signal
    \end{itemize}
    \item Ethernet (copper) cables: 
    \begin{itemize}
        \item Encoded as pulses of electricity 
    \end{itemize}
    \item Fiber cables:
    \begin{itemize}
        \item Encoded as pulses of light
    \end{itemize}
    \item Different physical properties so they may have different protocols
    \item Cabled connections
    \begin{itemize}
        \item DLL protocol = 802.3 Ethernet
        \item Point to point connections (exactly two devices)
        \begin{itemize}
            \item Fiber
            \begin{itemize}
                \item Full enclosed glass tubes with mirrored shielding
                \item Photons bounce along the cable until it reaches a detecor 
                \item Needs to be as straight as possible
            \end{itemize}
        \end{itemize}
        \begin{itemize}
            \item Copper / Twisted Pair
            \begin{itemize}
                \item Pair needed to complete the circuit
                \item Two electrical magnetic field generated positive and negatively charged respectively
                \item Field is powerful enough to corrupt data
                \item Electomagnetic inteference cancel each other out if they are close enough hence twisted 
            \end{itemize}
        \end{itemize}
    \end{itemize}
    \item Wireless connections
    \begin{itemize}
        \item DLL protocol = 802.11 WiFi
        \item Non-directional 
        \begin{itemize}
            \item All wireless devices go in every direction
            \item Only matters if the device is within the range
            \item Everyone in the recipient range of the device recieves the data
        \end{itemize}
        \item Encryption by default
        \begin{itemize}
            \item Encryption by default has speed cost
        \end{itemize}
        \item Encoded as radiowaves
    \end{itemize}
    \item Channel Types
    \begin{itemize}
        \item Simplex (unidirectionality)
        \item Duplex (biderectionality)
        \begin{itemize}
            \item Full Duplex (send and receive at the same time)
            \item Half Duplex (send or receive at any time only one receiver)
        \end{itemize}
        \item All cables can full duplex only restriction is cost
        \item All wireless operate at half duplex
        \item Modern devices operate at half duplex but extremely highspeed
        \begin{itemize}
            \item Operates on the scale of picoseconds
        \end{itemize}
        \item Feels like full duplex 
    \end{itemize}
    \item Hardware
    \begin{itemize}
        \item Each layer has a specific hardware devices
        \item Network Hub (Physical Layer, Historical)
        \begin{itemize}
            \item Solve the limitation of the cable
            \item Central hub that ideally connects all devices
            \item Act as an n-dimensional cables
            \item No CPU or memory, just a bunch of logic gates (reducing cost)
            \item Duplicates signal and sends the signal to all ports
            \item Problem arises when two or more devices send at the same time data easily corrupted
            \begin{itemize}
                \item This is called a collision domain
                \item If there is more than one sender none of the messages get through
            \end{itemize}
            \item Cannot update the hub so you have to update all the devices connected
            \item Carrier Sense Multiple Access
            \begin{itemize}
                \item Solves collission domain problem by sharing the network
                \begin{lstlisting}
                    function CSMA(message){
                        while(receiving){
                            wait();
                        }
                        send(message);
                    }
                \end{lstlisting}
                \item Downgrades connection to half duplex 
                \item Limitation is assuming everyone upgrading at the same time
                \item Devices with CSMA and devices with no CSMA do not play well together
                \item Device without CSMA can still corrupt transmitted data

            \end{itemize}
            \item CSMA With Collision Dection
            \begin{lstlisting}
                function CSMACD (message){
                    for(i = 0; i < message.length; i++){
                        while(receiving){
                            wait();;
                        }
                        send(messaage[i]);
                    }
                }
            \end{lstlisting}
        \end{itemize}
        \item Network Switch
        \begin{itemize}
            \item 
        \end{itemize}
    \end{itemize}
\end{itemize}

\section*{Class 9 \\ 02/27/2025}\label{sec:Class 9}
\begin{itemize}
    \item Topic 6: Data Link Layer
    \begin{itemize}
        \item Translation from digital bits into physical bits
        \item Hop to hop delivery between two directly connect adjacent nodes over a single transmission medium (can be wireless)
        \item Addressing:
        \begin{itemize}
            \item How to distinguish between different devices
            \item DLL (Data Link Layer): MAC Address
            \begin{itemize}
                \item A globally unique hardware identifier assigned to every network card
                \item No two MAC addresses should be the same
                \item 48 bits \(2^{48}\) = 281 trillion
                \item ex. 6E-3F-BB-22-A0-0F (6 pairs of hexadecimal numbers)
                \item First 3 pairs = Organizational Unique Identifier
                \item Last 3 pairs = Network Interface Component/Card
                \item MAC addresses managed by IEEE
                \item First 3 pairs are  assigned to a company
            \end{itemize}
            \item You can change the MAC address but in 99\% it doesn't make sense
            \item MAC address survive a single hop and only used to bring you one hop closer
            \item Cannot fool anyone more than 1 hop away from you
        \end{itemize}
    \item Protocols
    \begin{itemize}
        \item 802.3 Ethernet (wired connections): simple header and data protocol
        \item 802.11 WiFi (wireless): complex protocol
        \begin{itemize}
            \item Many (100s) of different types of messages
        \end{itemize}
        \item Underlying physical properties between the two are so different they need different protocols
        \item Ex. ethernet cable you just need to plug in but for wireless you need to select the correct device and put in a password
        \item Wireless needs a more complex protocol to accomodate its physical properties
        \item Ethernet has a faster protocol focused on throughput 
    \end{itemize}
    \item What is provided by the header?
    \begin{itemize}
        \item Preamble = alternating set of 0 and 1 used to provide synchronization between sender and receiver 
        \begin{itemize}
            \item Devices operate at different speeds but need to communicate
            \item SFD last byte of preamble contains a bunch of 1's to signal that the preamble is over
        \end{itemize}
        \item Destination MAC and Source MAC
        \begin{itemize}
            \item Lets us know who the sender is and who the receiver is 
            \item Each number = number of bytes
        \end{itemize}
        \item EtherType
        \begin{itemize}
            \item Tells us the network protocol used in the payload, most common:
            \item 0x0800 = IPv4
            \item 0x86DD = IPv6
            \item 0x0806 = ARP 
            \item 0x8100 = IPv4 and 802.1Q
        \end{itemize}
        \item IPv4: mainly used in the US, we didn't make the IP address field large enough 
        \item IPv6: used in addition, used in the rest of the world
        \item ARP: used to translate IP to MAC address
        \item IPv4 + 802.1Q: 802.1Q provides VLAN support 
        \item Provides logical seperation of networks where you are phsyically conencted to other devices
        \item Provides quality of service, refers to priority
        \item Allows us to assign higher priority to different devices
        \item Can make a difference in real time applications
        \item CRC/FCS 
        \begin{itemize}
            \item Frame Check Sequence
            \item Algorithm provides data integrity
            \item Ability to detect when data is corrupted while in transit
            \item Internally uses CRC32 hash algorithm to provide the property
            \item Takes variable sized input, fed into hash algorithm, received fixed length digest
        \end{itemize}
        \item Interframe = ensures messages dont collide
        \item Data Integrity vs Reliable delivery
        \begin{itemize}
            \item Data Integrity = Ability to detect when data is corrupted
            \item Reliable delivery = Constantly redeliver message until it is delivered
        \end{itemize}
        \item Ethernet itself does not provide reliable delivery
        \item Reliable delivery is not free, data integrity is more important
    \end{itemize}
    \end{itemize}
\end{itemize}

\end{document}