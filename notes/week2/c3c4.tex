\documentclass{article}

\usepackage[english]{babel}
\usepackage{listings}
\usepackage{color}
\usepackage{lipsum}
\usepackage{soul}


\definecolor{dkgreen}{rgb}{0,0.6,0}
\definecolor{gray}{rgb}{0.5,0.5,0.5}
\definecolor{mauve}{rgb}{0.58,0,0.82}

\lstdefinelanguage{JavaScript}{
  keywords={let, typeof, new, true, false, catch, function, return, null, catch, switch, var, if, in, while, do, else, case, break},
  keywordstyle=\color{blue}\bfseries,
  ndkeywords={class, export, boolean, throw, implements, import, this},
  ndkeywordstyle=\color{gray}\bfseries,
  identifierstyle=\color{black},
  sensitive=false,
  comment=[l]{//},
  morecomment=[s]{/*}{*/},
  commentstyle=\color{purple}\ttfamily,
  stringstyle=\color{red}\ttfamily,
  morestring=[b]',
  morestring=[b]'
}

\lstset{frame=tb,
  language=JavaScript,
  aboveskip=3mm,
  belowskip=3mm,
  showstringspaces=false,
  columns=flexible,
  basicstyle={\small\ttfamily},
  numbers=none,
  numberstyle=\tiny\color{gray},
  keywordstyle=\color{blue},
  commentstyle=\color{dkgreen},
  stringstyle=\color{mauve},
  breaklines=true,
  breakatwhitespace=true,
  tabsize=3
}

% Remember to update
\title{Week 2 Notes}
\date{}
\author{Ross Emile Aparece}

\begin{document}
\maketitle

% Remember to update
\section*{Class 3 \\ 02/04/2025}\label{sec:Class 3}
\begin{itemize}
    \item Standard syntax for an object is key value pairs
    \begin{lstlisting}
        let myfrac = {num:1, den:4};
    \end{lstlisting}
    \item Keys can be any type of string value but there are restrictions
    \item Do not need quotes for a property
    \item Values can be any data type
    \begin{lstlisting}
        let myfrac = {
            num:1, 
            den:4, 
            toDecimal: function(){return this.num / this.den}
            };
    \end{lstlisting}
    \item Functions are called using ()
    \begin{lstlisting}
        myfrac.toDecimal;
        returns f(){return this,num / this.den}

        myfrac.toDecimal();
        returns 0.25;
    \end{lstlisting}
    \item Deconstructing assignment
    \begin{lstlisting}
        let downloaded = {a:5, b:7, c:8, z:89};

        console.log(b);
        console.log(z);
    \end{lstlisting}
    \item Spread operator converst from container type to parameter list
    \begin{lstlisting}
        let arr1 = [1, 2, 3, 4 , 5];
        let arr2 = [8, 9, 10]

        [..arr1, 6, 7, ...arr2, 10, 11, 12,13]
        returns (13) [1, 2, 3, 4, 5, 6, 7, 8, 9, 10 , 11, 12, 13]
    \end{lstlisting}
    \item Informally removes the outermost braces and turns it into a comma seperated list
    \item Works with any container type, common use case:
    \begin{lstlisting}
        let downloaded = {a:5, b:7, c:8, z:89}
        let augmenteded = {...downloaded, d:100}
    \end{lstlisting}
    \item Spread operator cannot be used anywhere
    \begin{lstlisting}
        ...arr1 
        let arr1 = [1, 2, 3, 4 , 5];
        let arr2 = [8, 9, 10]

        //but this works:

        [...arr1] 
        let arr1 = [1, 2, 3, 4 , 5];
        let arr2 = [8, 9, 10]
    \end{lstlisting}
    \begin{itemize}
        \item Assume downloaded is a massive array
    \end{itemize}
    \begin{lstlisting}
        Math.max(downloaded);
        //Returns NaN because it wants a parameter list

        Math.max(...downloaded);
        //Converts the array to a parameter list allowing the function to work properly
    \end{lstlisting}
    \item Rest operator ... (Same symbol as spread operator)
    \begin{itemize}
        \item Context lets us determine the difference between the two
    \end{itemize}
    \item Converts from parameter list to array
    \item Scenario: assume download is a massive array
    \begin{lstlisting}
        let download = [5, 6, 7, 8, 9, 10, 11];
        let [first,second, ...r] = downloaded;

        console.log(first);
        console.log(second);

        console.log(r);
        //returns the array containing the rest of the numbers
    \end{lstlisting}
    \item Second use case:
    \begin{lstlisting}
        //Use a rest operator for an infinite amount of numberes
        function myMax(...n){
            let max_candidate = n[0];
            for(let i = 1; i < n.length; i++){
                if(n[i]< nax_candidate){
                    max_candidate = n[i];
                }
            }
            return max_candidate;
        }

        myMax(1, 2, 3, 4, 5, 6, 7, 8, 9);
        //Should return 9;
    \end{lstlisting}
    \begin{itemize}
        \item Allows a function to accept any amount of inputs
    \end{itemize}
    % Topic 2
    \item Topic 2 start
    \begin{lstlisting}
        let n1;
        n1 = 31;
        let n2;
        n2 = n1;
        n1 = 32;
        console.log(n2);
    \end{lstlisting}
    \begin{center}
        STACK
    \begin{tabular}{|c|c|c|}
        \hline ID & ADD & VAL  \\
        \hline\hline
        n1 & 0x01 & \st{undefined} \st{31} 32 \\
        n2 & 0x02 & \st{undefined} 31 \\
        \hline    
    \end{tabular}
    \end{center}
\end{itemize}

\section*{Class 4 \\ 02/06/2025}\label{sec:Class 4}
\begin{itemize}
    \item Shallow Copy example 2
    \begin{lstlisting}
        let f1 = {num: 3, den: 4, inverse:{num:4, den:3}}
        let f2 = {...f1};
        f1.num = 1;
        f1.den = 2;
        f1.inverse.num = 2;
        f1.inverse.den = 1;
        
        console.log(f2);
        //returns 3/4 and 2/1
    \end{lstlisting}
    \begin{itemize}
        \item We want 3/4 and 4/3
        \item Not a true copy hence SHALLOW copy
    \end{itemize}
    \begin{center}
        STACK
        \begin{tabular}{|c|c|c|} \hline ID & ADD & VAL \\
            \hline\hline 
            f1 & 0x01 & \st{undefined} 0xA2 \\
            f2 & 0x02 & \st{undefined} 0xA3  \\
            \hline
        \end{tabular}
    \end{center}
    \begin{center}
        HEAP
        \begin{tabular}{|c|c|}
            \hline ADD & VAL \\
            \hline\hline
            0xA1 & \{num:\st{4} 2, den:\st{3} 1\} \\
            0xA2 & \{num:\st{3} 1, den:\st{4} 2, inverse: 0xA1\} \\
            0xA3 & \{num:3, den: 4, inverse: 0xA1\} \\
            \hline
        \end{tabular}
    \end{center}
    \item True copy
    \begin{lstlisting}
        let f2 = structuredClone{1};
    \end{lstlisting}
    \begin{itemize}
        \item structuredClone should be used in the server environment
        \item However not every browser supports structureClone
    \end{itemize}
    \item How do we target browsers on an older version?
    \begin{lstlisting}
        let f2 = JSON.parse(JSON.stringify(f1));
    \end{lstlisting}
    \begin{itemize}
        \item Converts object to a JSON string and converts it back into an object
        \item Useful to target legacy systems
        \item Doesn't preserve a lot of data types
        \begin{itemize}
            \item Functions may be dropped
            \item Complex data types may be dropped
        \end{itemize}
    \end{itemize}
    \item Scopes:
    \begin{itemize}
        \item Global scope, function scope, block scope
        \item Lexical scope 
        \item Module scope (not discusssed)
    \end{itemize}
    \item Lexical is more of theory than actual scope
    \item Block scope will be used within this class
    \item Global Scope:
    \begin{itemize}
        \item Not recommended due to namespace pollution
        \item There is no global keyword
    \end{itemize}
    \begin{lstlisting}
        function myscope(){
            x = 5;
        }
        myscope();
        console.log(x);
    \end{lstlisting}
    \begin{itemize}
        \item x is a global variable in this scenario
        \item Not using let, var, const makes the variable global
    \end{itemize}
    \begin{lstlisting}
        function myscope(){
            var x;
            x = 5;
        }
        myscope();
        console.log(x);
    \end{lstlisting}
    \begin{itemize}
        \item First use of the variable decides if its global or not
    \end{itemize}
    \item Global variables are bad because namespace pollution where variable name is no longer available
    \item Makes it hard to maintain even if you are working alone
    \item There are some specific usecases: building a language, building a library, etc.
    \begin{itemize}
        \item If you are building for a developer and not an end user
    \end{itemize}
    \item Function Scope:
    \begin{itemize}
        \item Exists anywhere within the function
    \end{itemize}
    \begin{lstlisting}
        function myscope(){
            var x = 5;
        }
        myscope();
        console.log(x);
        //prints out undefined
    \end{lstlisting}
    \begin{lstlisting}
        function myscope(){
            console.log(x);
            var x = 5;
        }
        myscope();
    \end{lstlisting}
    \begin{lstlisting}
        function myscope(){
            console.log(x);
            //var x = 5;
        }
        myscope();
    \end{lstlisting}
    \begin{lstlisting}
        function myscope(){
            if (false) {
                var x = 5;
            }
            console.log(x);
        }
        myscope();
        //prints out undefined
    \end{lstlisting}
    \begin{lstlisting}
        function myscope(){
            if (false) {
                //var x = 5;
            }
            console.log(x);
        }
        myscope();
        //doesnt work
    \end{lstlisting}
    \begin{itemize}
        \item Variables defined in function scope have preprocessing 
        \item Moves the declaration to the top of the function
        \item The variable is accessable anywhere within the function
    \end{itemize}
    \item Block Scope (Recommended):
    \begin{itemize}
        \item Using let or const
    \end{itemize}
    \begin{lstlisting}
        function myscope(){
            if (true) {
                let var x = 5;
            }
            console.log(x);
            //x cannot be accessed outside the braces 
        }
        myscope();
    \end{lstlisting}
    \begin{itemize}
        \item Scoped within the braces 
        \item No hoisting at all
    \end{itemize}
    \begin{lstlisting}
        function myscope(){
            if (true) {
                let var x = 5;
                console.log(x);
            }
        }
        myscope();
    \end{lstlisting}
    \begin{itemize}
        \item You don't even need anything before braces
    \end{itemize}
    \begin{lstlisting}
        {
            let var x = 5;
            console.log(x);
        }
    \end{lstlisting}
    \begin{itemize}
        \item Not a real use case
    \end{itemize}
    \item Lexical scope:
    \begin{lstlisting}
        function a(){
            let data1 = 1;
            function b(){
                let data2 = 2;
                function c() {
                    let data3 = 3;
                    console.log(data1, data2, data3);
                }
            c();
            console.log(data1, data2);
            }
            b();
        }
        a();
    \end{lstlisting}
    \begin{itemize}
        \item You can access the ancestors variables
        \item Only works inner to outer
    \end{itemize}
    \item High Order Functions and Closures
    \begin{itemize}
        \item Either/or/and:
        \begin{itemize}
            \item Function that takes another function as an input
            \item Function that returns a function as output
        \end{itemize}
        \item Most common is map function
    \end{itemize}
    \begin{lstlisting}
        [1,2,3,4,5,6,7].map(someFunction);
    \end{lstlisting}
    \begin{itemize}
        \item Maps the function to each element of the array and returns a new array with the function applied to the elements
    \end{itemize}
    \begin{lstlisting}
        [1,2,3,4,5,6,7].map(x => x + 1);
    \end{lstlisting}
    \begin{itemize}
        \item Returns an array with each element + 1
        \item Create adder function
    \end{itemize}
    \begin{lstlisting}
        function create_adder(x){
            return function(y){
                return x + y;
            }
        }
        const add12 = create_adder(12);
        add12(10);
    \end{lstlisting}
    \begin{lstlisting}
        function create_adder(x){
            return function(y){
                return x + y;
            }
        }
        const add5 = create_adder(5);
        add12(100);
    \end{lstlisting}
    \begin{itemize}
        \item Sends back a function that binds x and has a floating variable y that we set
        \item We can have both a function that accepts a function as an input and returns a function as an output
    \end{itemize}
    \begin{lstlisting}
        function outer(a){
            let b = 20;
            let unused = 50;
            return function inner(c){
                let d = 40;
                return `${a}, ${b}, ${c}, ${d}`;
            }
        }
        let i = outer(10);
        let j = i(30);
        console.log(j);
    \end{lstlisting}
    \begin{itemize}
        \item Returns 10, 20, 30, 40
        \item We are accessing something outer when we are no longer within that scope
        \item Closure is the preservation of access to varaiable that another function (or block of code) depends on 
        \begin{itemize}
            \item Does not preserve the data
        \end{itemize}
    \end{itemize}
    \begin{lstlisting}
        return function inner(c){
            let d = 40;
            return `${a}, ${b}, ${c}, ${d}`;
        }
        Env:{a:0x03, b:0x05}
        //environmental variables that lets us maintain access that allows us to break scope
    \end{lstlisting}
    \begin{itemize}
        \item Closure ensures that the garbage collector does not delete those elements test
    \end{itemize}
\end{itemize}
\end{document}