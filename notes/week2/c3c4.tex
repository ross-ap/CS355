\documentclass{article}

\usepackage[english]{babel}
\usepackage{listings}
\usepackage{color}
\usepackage{lipsum}

\definecolor{dkgreen}{rgb}{0,0.6,0}
\definecolor{gray}{rgb}{0.5,0.5,0.5}
\definecolor{mauve}{rgb}{0.58,0,0.82}

\lstdefinelanguage{JavaScript}{
  keywords={let, typeof, new, true, false, catch, function, return, null, catch, switch, var, if, in, while, do, else, case, break},
  keywordstyle=\color{blue}\bfseries,
  ndkeywords={class, export, boolean, throw, implements, import, this},
  ndkeywordstyle=\color{gray}\bfseries,
  identifierstyle=\color{black},
  sensitive=false,
  comment=[l]{//},
  morecomment=[s]{/*}{*/},
  commentstyle=\color{purple}\ttfamily,
  stringstyle=\color{red}\ttfamily,
  morestring=[b]',
  morestring=[b]'
}

\lstset{frame=tb,
  language=JavaScript,
  aboveskip=3mm,
  belowskip=3mm,
  showstringspaces=false,
  columns=flexible,
  basicstyle={\small\ttfamily},
  numbers=none,
  numberstyle=\tiny\color{gray},
  keywordstyle=\color{blue},
  commentstyle=\color{dkgreen},
  stringstyle=\color{mauve},
  breaklines=true,
  breakatwhitespace=true,
  tabsize=3
}

% Remember to update
\title{Week 2 Notes}
\date{}
\author{Ross Emile Aparece}

\begin{document}
\maketitle

% Remember to update
\section*{Class 3 \\ 02/04/2025}\label{sec:Class 3}
\begin{itemize}
    \item Standard syntax for an object is key value pairs
    \begin{lstlisting}
        let myfrac = {num:1, den:4};
    \end{lstlisting}
    \item Keys can be any type of string value but there are restrictions
    \item Do not need quotes for a property
    \item Values can be any data type
    \begin{lstlisting}
        let myfrac = {
            num:1, 
            den:4, 
            toDecimal: function(){return this.num / this.den}
            };
    \end{lstlisting}
    \item Functions are called using ()
    \begin{lstlisting}
        myfrac.toDecimal;
        returns f(){return this,num / this.den}

        myfrac.toDecimal();
        returns 0.25;
    \end{lstlisting}
    \item Deconstructing assignment
    \begin{lstlisting}
        let downloaded = {a:5, b:7, c:8, z:89};

        console.log(b);
        console.log(z);
    \end{lstlisting}
    \item Spread operator converst from container type to parameter list
    \begin{lstlisting}
        let arr1 = [1, 2, 3, 4 , 5];
        let arr2 = [8, 9, 10]

        [..arr1, 6, 7, ...arr2, 10, 11, 12,13]
        returns (13) [1, 2, 3, 4, 5, 6, 7, 8, 9, 10 , 11, 12, 13]
    \end{lstlisting}
    \item Informally removes the outermost braces and turns it into a comma seperated list
    \item Works with any container type, common use case:
    \begin{lstlisting}
        let downloaded = {a:5, b:7, c:8, z:89}
        let augmenteded = {...downloaded, d:100}
    \end{lstlisting}
    \item Spread operator cannot be used anywhere
    \begin{lstlisting}
        ...arr1 
        let arr1 = [1, 2, 3, 4 , 5];
        let arr2 = [8, 9, 10]

        //but this works:

        [...arr1] 
        let arr1 = [1, 2, 3, 4 , 5];
        let arr2 = [8, 9, 10]
    \end{lstlisting}
    \begin{itemize}
        \item Assume downloaded is a massive array
    \end{itemize}
    \begin{lstlisting}
        Math.max(downloaded);
        //Returns NaN because it wants a parameter list

        Math.max(...downloaded);
        //Converts the array to a parameter list allowing the function to work properly
    \end{lstlisting}
    \item Rest operator ... (Same symbol as spread operator)
    \begin{itemize}
        \item Context lets us determine the difference between the two
    \end{itemize}
    \item Converts from parameter list to array
    \item Scenario: assume download is a massive array
    \begin{lstlisting}
        let download = [5, 6, 7, 8, 9, 10, 11];
        let [first,second, ...r] = downloaded;

        console.log(first);
        console.log(second);

        console.log(r);
        //returns the array containing the rest of the numbers
    \end{lstlisting}
    \item Second use case:
    \begin{lstlisting}
        //Use a rest operator for an infinite amount of numberes
        function myMax(...n){
            let max_candidate = n[0];
            for(let i = 1; i < n.length; i++){
                if(n[i]< nax_candidate){
                    max_candidate = n[i];
                }
            }
            return max_candidate;
        }

        myMax(1, 2, 3, 4, 5, 6, 7, 8, 9);
        //Should return 9;
    \end{lstlisting}
    \begin{itemize}
        \item Allows a function to accept any amount of inputs
    \end{itemize}
    % Topic 2
    \item Topic 2 start
    \begin{lstlisting}
        let n1;
        n1 = 31;
        let n2;
        n2 = n1;
        n1 = 32;
        console.log(n2);
    \end{lstlisting}
\end{itemize}

\section*{Class 4 \\ 02/06/2025}\label{sec:Class 4}


\end{document}