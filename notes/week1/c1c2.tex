\documentclass{article}

\usepackage[english]{babel}
\usepackage{listings}
\usepackage{color}
\usepackage{lipsum}

\definecolor{dkgreen}{rgb}{0,0.6,0}
\definecolor{gray}{rgb}{0.5,0.5,0.5}
\definecolor{mauve}{rgb}{0.58,0,0.82}

\lstdefinelanguage{JavaScript}{
  keywords={let, typeof, new, true, false, catch, function, return, null, catch, switch, var, if, in, while, do, else, case, break},
  keywordstyle=\color{blue}\bfseries,
  ndkeywords={class, export, boolean, throw, implements, import, this},
  ndkeywordstyle=\color{darkgray}\bfseries,
  identifierstyle=\color{black},
  sensitive=false,
  comment=[l]{//},
  morecomment=[s]{/*}{*/},
  commentstyle=\color{purple}\ttfamily,
  stringstyle=\color{red}\ttfamily,
  morestring=[b]',
  morestring=[b]'
}

\lstset{frame=tb,
  language=JavaScript,
  aboveskip=3mm,
  belowskip=3mm,
  showstringspaces=false,
  columns=flexible,
  basicstyle={\small\ttfamily},
  numbers=none,
  numberstyle=\tiny\color{gray},
  keywordstyle=\color{blue},
  commentstyle=\color{dkgreen},
  stringstyle=\color{mauve},
  breaklines=true,
  breakatwhitespace=true,
  tabsize=3
}

% Remember to update
\title{Week 1 Notes: JavaScript Basics}
\date{}
\author{Ross Emile Aparece}

\begin{document}
\maketitle

% Remember to update
\section*{Class 1 \\ 01/28/2025}\label{sec:Class 1}

\pagebreak
\section*{Class 2 \\ 01/30/2025}\label{sec:Class 2}
\begin{itemize}
  \item JavaScript is backwards compatible, have to live with the mistakes in all previouse versions
  \item JavaScript interpreter is convienent for prototyping
  \item console.log();
    \begin{lstlisting}
      console.log("Hello World!");
      //prints "Hello World!"
      console.log(5 > 2);
      //prints true
    \end{lstlisting}
    \begin{itemize}
      \item For arrays JS will return the size of the array and the contents of the array
    \end{itemize}
  \item console.table();
    \begin{itemize}
      \item returns an xy-table for the array
    \end{itemize}
  \item Data Types: Primitives and objects
    \begin{itemize}
      \item 7 Primitives; everything else is an objects
      \begin{itemize}
        \item number (64 bit IEEE 754 double precision)
        \begin{itemize}
          \item Range: \(10^{127}\) to \(1/10^{127}\)
          \item Loses precision: \(0.1 + 0.2 = 0.30000000000000004\)
          \item Better to avoid using number and stick with int
          \item Stay within max\_safe\_int and min\_safe\_int
        \end{itemize}
        \item string
        \begin{itemize}
          \item Same string as in Java  
          \item Can use " \ " or ' \ '
        \end{itemize}
        \item boolean
        \begin{itemize}
          \item True/False
        \end{itemize}
        \item undefined
        \item null
        \begin{itemize}
          \item Similar to undefined but can tell apart
          \item Undefined when you declare a variable but not assign a value
          \item Null is an explicit declaration that a value does not exist
          \item Undefined represent a temporary value
        \end{itemize}
        \item Symbol (Not discussed)
        \begin{itemize}
          \item Used for complete uniqueness, ex.\ primary keys
        \end{itemize}
        \item BigInt (Not discussed)
        \begin{itemize}
          \item Used for arbitrarily large numbers
        \end{itemize}
      \end{itemize}
      \item Primitives cannot be decomposed but objects can
      \item Objects:
      \begin{itemize}
        \item arrays
        \begin{itemize}
          \item Objects with numerical indeces
        \end{itemize}
        \item functions
        \begin{itemize}
          \item First class variables
          \item Treated no different than any other data type
        \end{itemize}
        \item date/times
      \end{itemize}
    \end{itemize}
    \item JavaScript uses syntactic sugar
    \begin{itemize}
      \item Object Wrapper
      \begin{itemize}
        \item All primitivies have a mirrored object versions
        \item JavaScript will casts the primitive to its equivalent object version
      \end{itemize}
    \end{itemize}
    \item There are 3 ways to declare functions
    \begin{itemize}
      \item Function Declaration
      \begin{lstlisting}
        function sum(a,b){
          return a+b;
        }
        sum(3,5);
        //returns 8
      \end{lstlisting}
      \item Function Expression
      \begin{lstlisting}
        let sum = function add_two(a,b){
          return a + b;
        }
        sum(10,10);
        //returns 20
      \end{lstlisting}
      \item Arrow Functions
      \begin{itemize}
        \item If a function can be represented in a single line use arrow functions
        \item Intoduced in 2015 to shorten function expressions
      \end{itemize}
      \begin{lstlisting}
        let sum = (a,b) => a + b;
        sum(16, 8);
        //returns 24
      \end{lstlisting}
      \begin{itemize}
        \item If single parameter you can skip the parenthesis
      \end{itemize}
      \begin{lstlisting}
        let sq = x => x * x;
        sum(5, 5);
        //returns 25
      \end{lstlisting}
      \begin{itemize}
        \item If zero parameters the () must be brought back
      \end{itemize}
      \begin{lstlisting}
        let hello = () => 'hello';
        hello();
        //returns hello
      \end{lstlisting}
      \begin{itemize}
        \item Multiline arrow functions are not recommended to use
        \begin{itemize}
          \item Brings back curly braces and the return statement, losing a lot of the usefulness of an arrow function
        \end{itemize}
      \end{itemize}
      \begin{lstlisting}
        let sum_of_square = (a,b) => {
          let aa = aa;
          let bb = bb;
          return aa + bb;
      };
      \end{lstlisting}
    \end{itemize}

    \item Constructor Functions
    \begin{itemize}
      \item Input number i: 6
      \item Output object employee\{eid:i\}
    \end{itemize}
    \begin{lstlisting}
      function new_emp(i){
        return {eid:i};
      }
      new_emp(5);
      //returns eid:5

      let new_emp_arr = (i) => {eid:i};
      //returns undefined
    \end{lstlisting}
    \begin{itemize}
      \item Undefined so we have no idea whats wrongs
      \item Returns undefined ebcause it matches it with a multiline arrow function
      \item Solution:
    \end{itemize}
    \begin{lstlisting}
      let new_emp_a = (i) => ({eid:i});
      new_emp_a(7);
    \end{lstlisting}
\end{itemize}
\end{document}


