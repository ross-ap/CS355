\documentclass{article}

\usepackage[english]{babel}
\usepackage{listings}
\usepackage{color}
\usepackage{lipsum}

\definecolor{dkgreen}{rgb}{0,0.6,0}
\definecolor{gray}{rgb}{0.5,0.5,0.5}
\definecolor{mauve}{rgb}{0.58,0,0.82}

\lstdefinelanguage{JavaScript}{
  keywords={let, typeof, new, true, false, catch, function, return, null, catch, switch, var, if, in, while, do, else, case, break},
  keywordstyle=\color{blue}\bfseries,
  ndkeywords={class, export, boolean, throw, implements, import, this},
  ndkeywordstyle=\color{darkgray}\bfseries,
  identifierstyle=\color{black},
  sensitive=false,
  comment=[l]{//},
  morecomment=[s]{/*}{*/},
  commentstyle=\color{purple}\ttfamily,
  stringstyle=\color{red}\ttfamily,
  morestring=[b]',
  morestring=[b]'
}

\lstset{frame=tb,
  language=JavaScript,
  aboveskip=3mm,
  belowskip=3mm,
  showstringspaces=false,
  columns=flexible,
  basicstyle={\small\ttfamily},
  numbers=none,
  numberstyle=\tiny\color{gray},
  keywordstyle=\color{blue},
  commentstyle=\color{dkgreen},
  stringstyle=\color{mauve},
  breaklines=true,
  breakatwhitespace=true,
  tabsize=3
}

% Remember to update
\title{Week 3: Asynchronous Programming}
\date{}
\author{Ross Emile Aparece}

\begin{document}
\maketitle

% Remember to update
\section*{Class 5 \\ 02/11/2025}\label{sec:Class 5}
\begin{lstlisting}
    function not(boolean_func){
        return function(...x){
            return !boolean_func(...x);
        }
    }

    let is_even = x -> x % 2 === 0;
    let is_odd = not(is_even);
\end{lstlisting}

\begin{lstlisting}
    {
        //Loop A
        let i;
        for(i = 0; i < 10; i++){
            void setTimeout(() -> console.log(i), 3000);
        }
        //after 3 seconds run the function that prints i
        //prints 10 not 0 to 9
    }

    {
        //Loop B
        for(let i = 0; i < 10; i++){
            void setTimeout(() -> console.log(i), 3000);
        }
        //Actually prints 0 to 9
    }
\end{lstlisting}
\begin{itemize}
    \item Inside asynchronous programming these two loops are not the same
    \begin{itemize}
        \item For loop B on the stack it redeclares i and is all done through closures
    \end{itemize}
    \begin{lstlisting}
        //blocking 
        x = download(File)

        Use(x)

        print(hello);
    \end{lstlisting}
    \item We have latency in the webdev environment which cannot be solved
    \item print(hello) cannot be used until both x is done downloading and the Use function is done
    \begin{lstlisting}
        for(let i = 0; i < 5000000000000; i++){
            //blank intentionally
        }
        console.log("Finished!");
    \end{lstlisting}
    \item JS is single threaded so it can only do one thing at once
    \item Cannot interact with the webpage until the loop is done
    \item User expects a multitask environment
    \item Threaded model:
    \begin{itemize}
        \item Progrommer had a problem, so he used thread. Then hhade tprobwolems.
    \end{itemize}
    \item You dont get to tell the OS when a thread takes problem. Threads can run over each other.
    \item Async Model:
    \begin{itemize}
        \item Under the hood threads are still used but we are not allowed to access them.
        \item "I dont know how long this takes but when it finishes give v8 this callback function to run."
    \end{itemize}
    \begin{lstlisting}
        asynchronous_task(...data, callback);
    \end{lstlisting}
    \item Async Architecture (Top Down)
    \begin{itemize}
        \item Node.js Application
        \item Node.js API (JavaScript)
        \item Node.js Bindings (JavaScript to C/C++) Node.js Library C/C++ Addons
        \item V8 (JS Engine) LibUv(Library) c-ares llhttp / htpp-parser open-ssl zlib
        \item OS
    \end{itemize}
    \item Battle tested applications written in c/c++ 
    \item Bindings and C/C++ addons allows Node.js to be implemented on old proprietary programs
    \item LibUv:
    \begin{lstlisting}
        \item Provides access to multithreaded capabilities
    \end{lstlisting}
    \begin{lstlisting}
        //f01.txt
        Hello World

        //callback.js
        const fs = require("fs");
        fs.readFile("input/f01.txt", "utf8", do_after_reading);

        //err and data are output parameters for the callback function
        function do_after_reading(err, data){
            if(err){
                console.log(`Error reading file`');
            } else {
                console.log(`Finished Reading File`', data.length);
            }
        }
        console.log("Hello World");
    \end{lstlisting}
\end{itemize}

\pagebreak
\section*{Class 6 \\ 02/13/2025}\label{sec:Class 6}


\end{document}