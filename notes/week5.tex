\documentclass{article}

\usepackage[english]{babel}
\usepackage{listings}
\usepackage{color}
\usepackage{lipsum}

\definecolor{dkgreen}{rgb}{0,0.6,0}
\definecolor{gray}{rgb}{0.5,0.5,0.5}
\definecolor{mauve}{rgb}{0.58,0,0.82}

\lstdefinelanguage{JavaScript}{
  keywords={let, typeof, new, true, false, catch, function, return, null, catch, switch, var, if, in, while, do, else, case, break},
  keywordstyle=\color{blue}\bfseries,
  ndkeywords={class, export, boolean, throw, implements, import, this},
  ndkeywordstyle=\color{darkgray}\bfseries,
  identifierstyle=\color{black},
  sensitive=false,
  comment=[l]{//},
  morecomment=[s]{/*}{*/},
  commentstyle=\color{purple}\ttfamily,
  stringstyle=\color{red}\ttfamily,
  morestring=[b]',
  morestring=[b]'
}

\lstset{frame=tb,
  language=JavaScript,
  aboveskip=3mm,
  belowskip=3mm,
  showstringspaces=false,
  columns=flexible,
  basicstyle={\small\ttfamily},
  numbers=none,
  numberstyle=\tiny\color{gray},
  keywordstyle=\color{blue},
  commentstyle=\color{dkgreen},
  stringstyle=\color{mauve},
  breaklines=true,
  breakatwhitespace=true,
  tabsize=3
}

% Remember to update
\title{Topic 5: Physical Layer}
\date{}
\begin{document}

% Remember to update
\section*{Class 10 \\ 03/04/2025}\label{sec:Class 10}
DLL
\begin{itemize}
    \item  Hardware Device
    \begin{itemize}
        \item Network Switch
        \item Act as an n-dimensional cable
        \item Added a CPU and Memory
    \end{itemize}
    \item Network switch obseleted the network hub
    \item Memory allows the network switch to queue up messages
    \begin{itemize}
        \item Ex. A and C try to send to D at the same time
        \item A sends first and when done C starts sending
        \item Can alternate the data 
    \end{itemize}
    \item Switch can detect the header and make the correct decisions
    \item Switch is a passive device 
    \begin{itemize}
        \item Will not send messages out on its own
        \item Only fowards messages out on behalf of other devices
        \item How does the switch know about the topology of the network?
        \begin{itemize}
            \item Switch knows it is connected to something but doesn't know what they are 
            \item Uses the Switch Device Discovery Algorithm
        \end{itemize}
    \end{itemize}
\end{itemize}
Switch Device Discovery Algorithm
\begin{itemize}
    \item Upon receiving a message, record {SRC MAX + PHYS PORT} into FIB 
    \item Check {DEST MAC} for a matching record exists in FIB, FOWARDS message out on matching {PHYSICAL PORT} 
    
    else (if no record exists) fallback to hub mode, foward message to all other ports, except receiving port
    \item Nothing else happens
\end{itemize}
FIB = Forwarding Information Base 
\begin{itemize}
    \item Over time the network switch learns where the devices are located and records their MAC addresses and their respective port
    \item Will eventually map out the entire network
\end{itemize}
Switches do not have MAC addresses, doesn't need it. \\ 
Cannot assume that the last device resides on the last port. \\
Experiation field used to ensure that the memory is not monopolized by a MAC address that no longer exists. \\
Header remains the same when transmitted through multiple switches. \\
Two or more devices can be mapped to the same port. \\ \\
WiFi
\begin{itemize}
    \item Ethernet has a simple data and header protocol
    \item WiFi needs a complex protocol because of its physical properties
    \item Most of the messages are helper messages that deal with encryption, ensuring that you can connect, etc.
\end{itemize}
3 Major Categories:
\begin{itemize}
    \item Management Frames
    \item Control Frames
    \item Data Frames (Traditional data and header)
\end{itemize}
Management Frame:
\begin{itemize}
    \item Connection and disconnection with a particular network
    \item Beacon 
    \begin{itemize}
        \item Router constantly emits a beacon sending information for a device to connect
    \end{itemize}
    \item Probe
    \begin{itemize}
        \item Probing for a network that was connected to previously, to reconnect to it
    \end{itemize}
    \item Authentication
    \item Association
    \begin{itemize}
        \item After connection, allows us to interact with the network to get network resources or provide resources to other devices through the network
    \end{itemize}
\end{itemize}
Control Frames
\begin{itemize}
    \item Deals with the lossy nature of WiFi
    \item Acknowledgement
    \begin{itemize}
        \item WiFi does provides reliable delivery 
        \item Sends an acknowledgement message
        \item If ethernet sends an acknowledgement the time it would take would double
    \end{itemize}
    \item Block Ack 
    \begin{itemize}
        \item Acknowledge multiple messages with a single acknowledgement
    \end{itemize}
    \item RTS + CTS 
    \begin{itemize}
        \item Request to send, clear to send
        \item Solves hidden node problem
        \item Ugly solution in contrast to CSMA
        \begin{itemize}
            \item Sender sends request to send 
            \item Receiver sends clear to send 
            \begin{itemize}
                \item Signals to devices outside of the senders range to wait and switch to RTC + CTS
            \end{itemize}
            \item CSMA is faster, the extra messages cost bandwith and doesn't send actual data 
        \end{itemize}
    \end{itemize}
\end{itemize}
\end{document}